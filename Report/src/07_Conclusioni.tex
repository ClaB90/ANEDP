%\chapter{Conclusioni}
\section{Conclusioni e Lavori Futuri}
\label{chap:Conclusion}
Entrambi gli algoritmi implementati confermano i risultati teorici per l'ordine di convergenza dell'errore nei problemi di controllo ottimo parabolico se viene utilizzato uno schema di Petrov Galerkin. Per riscontrare questi esiti è necessario che l'errore in tempo non raggiunga la dimensione dell'errore in spazio. Inoltre la griglia temporale considerata deve avere un numero sufficiente di nodi per approssimare eventuali punti di non derivabilità nella funzione esatta.
\par
Lavori futuri potrebbero vertere sull'analisi dell'estensione dei teoremi di semi-newton nel caso per i problemi di controllo ottimo parabolici. 
\par
Nel codice potrebbe essere introdotta la lettura e scrittura su file per le soluzioni del problema di stato ed aggiunto in modo da ridurre il rischio di problemi di memoria, non riscontrati nei casi trattati da questo lavoro. 
\par
Il codice sviluppato permette l'utilizzo di diversi spazi funzionali per il problema di stato e di aggiunto nel caso di punto fisso ma non di semi-Newton. Il metodo di Petrov-Galerkin in spazio potrebbe essere introdotto anche per il secondo algoritmo.