%\chapter{Analisi del problema discreto}
\section{Analisi del problema discreto}
\label{chap:Discontinuos}

\subsection{Semidiscretizzazione temporale}
%\subsubsection{Caratteristiche dello schema numerico}
La sezione inizia introducendo la griglia e gli spazi funzionali utilizzati per la semidiscretizzazione temporale dei problemi di stato e aggiunto.

Si partizioni l'intervallo $[0,T)$ in sottointervalli $I_m=[t_{m-1},t_m)$, dove $0=t_0<t_1<\dots<t_M=T$ sono punti appartenenti al segmento $[0,T)$. Inoltre si affianchi a tale griglia una seconda partizione dell'intervallo $[0,T)$, fondamentale per ottenere i risultati di convergenza dei problemi semidiscretizzati, detta \textit{griglia duale}; essa è composta dagli intervalli $I^*_m=[t^*_{m-1},t^*_m)$, con $0=t^*_0<t^*_1<\dots<t^*_M=T$ e $t^*_m=\frac{t_{m-1}+t_m}{2}\quad\text{per $m=1,\dots,M$}$. Riassumendo si può contare su due suddivisioni dell'intervallo $[0,T)$ tali che $[0,T)=\bigcup^{M+1}_{m=1} I_m$ e $[0,T)=\bigcup^{M+1}_{m=1} I^*_m$.

Chiaramente la griglia primale avrà un parametro che ne descrive ``l'accuratezza'' che per noi sarà 
\[
k=\max_{1\le m\le M} k_m,\quad k_m=t_m-t_{m-1}\quad\text{per $m=1,\dots,M$}.
\] 

Per quanto riguarda l'ambientazione funzionale definiamo 
\begin{gather*}
P_k:=\Bigl\{v\in C([0,T],H^1_0(\Omega))\Bigl| v|_{I_{m}}\in \mathcal{P}_1(I_m,H^1_0(\Omega))\Bigr\}\hookrightarrow W(I),\\
P^*_k:=\Bigl\{v\in C([0,T],H^1_0(\Omega))\Bigl| v|_{I^*_{m}}\in \mathcal{P}_1(I^*_m,H^1_0(\Omega))\Bigr\}\hookrightarrow W(I),\\  
  Y_k:=\Bigl\{v:[0,T]\to H^1_0(\Omega)\Bigl| v|_{I_{m}}\in \mathcal{P}_0(I_m,H^1_0(\Omega))\Bigr\}\hookrightarrow W(I).
\end{gather*}
Qui, $\mathcal{P}_i(J,H^1_0(\Omega)), J\subset \overset{-}{I}, i\in\{0,1\}$, denota l'insieme dei polinomi in tempo al più di grado $i$ sull'intervallo $J$ a valori in $H^1_0(\Omega)$.


Nei risultati che si menzioneranno in seguito si farà spesso uso dei seguenti operatori di interpolazione:
\begin{enumerate}
\item $\mathcal{P}_{Y_{k}}:L^2(I,H^1_0(\Omega))\to Y_k$
\[
\mathcal{P}_{Y_{k}} v|_{I_{m}}:=\frac{1}{k_{m}} \int^{t_{m}}_{t_{m-1}}vdt\quad \text{for $m=1,\dots,M$},\quad\text{e}\quad \mathcal{P}_{Y_{k}}v(T):=0
\]
\item $\Pi_{Y_{k}}:C([0,T],H^1_0(\Omega))\to Y_k$
\[
\Pi_{Y_{k}} v|_{I_{m}}:=v(t^*_m)\quad\text{per $m=1,\dots,M$},\quad \Pi_{Y_{k}} v(T):=v(T)
\]
\item $\pi_{P^*_k}:C([0,T],H^1_0(\Omega))\cup Y_k\to P^*_k$
\begin{gather*}
\pi_{P^*_k} v|_{I^*_1\cup I^*_2}:=v(t^*_1)+\frac{t-t^*_1}{t^*_2-t^*_1}(v(t^*_2)-v(t^*_1)),\\
\pi_{P^*_k} v|_{I^*_m}:=v(t^*_{m-1})+\frac{t-t^*_{m-1}}{t^*_{m}-t^*_{m-1}}(v(t^*_m)-v(t^*_{m-1})),\quad\text{per $m=3,\dots,M-1$},\\
\pi_{P^*_k} v|_{I^*_M\cup I^*_{M+1}}:=v(t^*_{M-1})+\frac{t-t^*_{M-1}}{t^*_M-t^*_{M-1}}(v(t^*_M)-v(t^*_{M-1})).
\end{gather*}
\end{enumerate} 


\subsection{Semidiscretizzazione del problema di stato}
Uno degli elementi di novità del lavoro \ref{MAIN} è dato proprio dallo schema adottato per la discretizzazione in tempo dell'equazione di stato, che è totalmente variazionale di tipo Petrov-Galerkin. Verranno infatti utilizzati uno spazio di funzioni costanti a tratti per la soluzione e continue lineari a tratti per le test. 

Supponendo di estendere la forma bilineare $A$ di \cite{MAIN} in una mappa $A:W(I)\cup Y_k\times\to\R$, si tratta dunque di risolvere il seguente problema:

dati $(f,\kappa)\in L^2(I,H^{-1}(\Omega))\times L^2(\Omega)$, trovare $y_k\in Y_k$ tale che 
\begin{equation}
\label{eqn:stato}
A(y_k,v_k)=\int^{T}_{0} \Braket{f(t),v_k(t)}_{H^{-1}(\Omega)H^1_0(\Omega)}dt + (\kappa,v_k(0))_{L^2(\Omega)}\quad \forall v_k\in P_k.
\end{equation}
Data la scelta dello spazio $Y_k$ per la soluzione, essa può essere espressa come 
\[
y_k=\alpha_{M+1}\chi_{\{T\}} + \sum^M_{i=1} \alpha_i\chi_{I_i},\quad \alpha_i\in H^1_0(\Omega)\quad\text{per $i=1,\dots,M+1$},
\]
e attraverso facili calcoli si arriva alla formulazione del problema~\eqref{eqn:stato} in termini delle funzioni $\alpha_i$:
\begin{enumerate}
\item Trovare $\alpha_1$ tale che
\begin{equation}
\label{eqn:stato1}
\frac{1}{t_1-t_0}(\alpha_1-\kappa,g)_{L^2(\Omega)} + \frac{1}{2}a(\alpha_1,g)=\frac{1}{2}(f(0),g)_{L^2(\Omega)}\quad \forall g\in H^1_0(\Omega),
\end{equation}
\item Trovare $\alpha_i,i=2,\dots,M$ tale che 
\begin{equation}
\label{eqn:stato2}
 \frac{1}{t_i-t_{i-1}}(\alpha_i-\alpha_{i-1},g)_{L^2(\Omega)} + \frac{1}{2}a(\alpha_i+\alpha_{i+1},g)=\frac{1}{2}(f(t_{i-1}),g)_{L^2(\Omega)}\quad \forall g\in H^1_0(\Omega),
 \end{equation}
 \item Trovare $\alpha_{M+1}$ tale che 
\begin{equation}
\label{eqn:stato3}
\frac{1}{t_{M}-t_{M-1}}(\alpha_{M+1}-\alpha_M,g)_{L^2(\Omega)} + \frac{1}{2}a(\alpha_M,g)=\frac{1}{2}(f(t_{M}),g)_{L^2(\Omega)}\quad \forall g\in H^1_0(\Omega),
\end{equation}
\end{enumerate}
Alla luce di questa seconda formulazione si evince che lo schema di Petrov-Galerkin introdotto è equivalente ad una variante del metodo di Crank-Nicolson con primo e ultimo passo temporale di Rannacher, e sempre dalla seconda formulazione risulta chiaro che il problema semidiscretizzato ammette unica soluzione $y_k\in Y_k$.

A questo punto vengono enunciati senza dimostrazione alcuni risultati tratti da \cite{MAIN} che garantiscono la stabilità e convergenza del metodo di Petrov-Galerkin. Per la stabilità:
 \begin{lemma} 
 \label{stab:stato}
 Sia $y_k\in Y_k$ la soluzione di~\eqref{eqn:stato} con $f\in L^2(I,L^2(\Omega))$ e $\kappa\in L^2(\Omega)$ assegnati. Allora esiste una costante $C>0$ indipendente dal parametro di griglia $k$ tale che
 \[
 \norma{y_k}_I\le C(\norma{f}_I + \norma{\kappa}_{L^2(\Omega)})
 \]
\end{lemma}
 Passando ora all'analisi di convergenza è normale aspettarsi un'approssimazione di ordine $\mathcal{O}(k)$, poiché la $y_k$ é costante a tratti in tempo; tuttavia la proiezione di $y_k$ tramite l'operatore $\pi_{P^*_k}$ permette di ottenere stime di ordine $\mathcal{O}(k^2)$. Infatti,
 \begin{lemma} 
 \label{conv:stato}
 Siano $f\in H^1(I,L^2(\Omega)),f(0)\in H^1_0(\Omega)$ e $\kappa\in H^1_0(\Omega)$ con $\Delta\kappa\in H^1_0(\Omega)$ e siano $y,y_k$ le soluzioni dei problemi \ref{eq:201} e~\eqref{eqn:stato} con dati $(f,\kappa)$. Allora vale
 \[
 \norma{\pi_{P^*_k}y_k-y}_I\le Ck^2(\norma{\partial^2_t y}_I + \norma{\partial^2_t \Delta y}_I)
 \]
 \end{lemma}
 
\subsubsection{Passo di Rannacher}
In questa sezione si espone brevemente cosa significhi un passo di Rannacher all'interno di uno schema numerico per la soluzione di EDP.

Come è noto i problemi lineari continui di diffusione esibiscono il cosiddetto `` effetto regolarizzante '' ovvero che la soluzione dell'equazione del calore in assenza di coefficienti discontinui è di classe $ C^{\infty} $ anche se il dato iniziale è una distribuzione. Le modifiche di uno schema numerico per equazioni paraboliche attraverso alcuni passi di Rannacher servono proprio a restituire questa proprietà anche al problema discreto.

Per essere più concreti, si consideri la seguente equazione lineare di diffusione-convezione
\begin{gather}
\label{heat}
\partial_t u + Au=f \ \text{in}\ \Omega\times(0,\infty),  \\
u(0)=u^0 \ \text{in}\ \Omega, \\
u=d \ \text{in} \ \partial\Omega\times (0,\infty).
\end{gather}

dove $ A $ è un operatore differenziale ellittico e sono state scelte condizioni al bordo di solo Dirichlet per semplicità. Si supponga ora che ~\eqref{heat} venga discretizzato con elementi finiti in spazio di modo che la semidiscretizzazione conduca al seguente sistema di equazioni differenziali ordinarie
\begin{gather}
\label{heatsd}
\partial_t u_h + A_hu_h=F_h,\ t\in(0,\infty) \\
u_h(0)=u^0_h.
\end{gather}
Si tratta dunque di affrontare, trascurando il termine noto, un sistema del tipo
\begin{gather}
\label{pbmod}
\frac{\partial\psi}{\partial t}=H(t)\psi(t)  \\
\psi(0)=\psi_0
\end{gather}
la cui soluzione formale è $ \psi(t)=\psi_0\exp(\int^t_0 H(s)ds) $. Ora si può pensare, al fine di ricavare uno schema numerico, di sostituire l'esponenziale con una sua approssimante di Padè di ordine $ (\nu,\mu) $ e di supporre l'operatore $ H(t) $ costante in tempo; scegliendo per esempio $ \nu=1, \ \mu=1$ si ottiene lo schema di Crank-Nicolson, con la premura di aggiungere un termine che tenga conto dell'integrazione in tempo del termine noto. Il problema riscontrato dai metodi con $ \nu=\mu $ (schemi di Padè diagonali) è la mancanza di assoluta stabilità. Questa proprietà è invece verificata da algoritmi con $ \mu<\nu $ (schemi di Padè sub-diagonali). Ci si aspetta quindi che gli schemi di Padè diagonali propaghino maggiormente gli errori locali dovuti alle discontinuità dei dati rispetto a quelli sub-diagonali. 
L'idea del passo di Rannacher consiste, qualora si scegliesse di utilizzare uno schema di Padè diagonale, nella sostituzione dei passi in corrispondenza di irregolarità nel termine noto o nelle condizioni al bordo con altri tipici di schemi sub-diagonali. In effetti si dimostra che la sostituzione ottimale per uno schema di ordine $ (\mu,\mu) $ si compie attraverso passi sub-diagonali $ (\mu-1,\mu) $. 

\subsection{Semidiscretizzazione del problema aggiunto}
Anche il problema aggiunto viene affrontato con uno schema in tempo di tipo Petrov-Galerkin, dove però vengono scambiati gli spazi funzionali delle soluzioni e delle test. Poiché tale scelta conduce ad un metodo di Crank-Nicolson all'indietro, la sua analisi è standard. Si riportano qui di seguito la formulazione e risultati principali.

Problema aggiunto semidiscretizzato in tempo:

Dato $h\in L^2(I,H^{-1}(\Omega))$ trovare $p_k\in P_k$ tale che 
\begin{equation}
\label{eqn:agg}
A(\tilde{y},p_k)=\int^T_0 \Braket{h(t),\tilde{y}(t)}_{H^{-1}(\Omega)H^1_0(\Omega)}dt\quad\forall\tilde{y}\in Y_k
\end{equation}
Se si scrive $p_k$ nella forma 
\[
p_k(t)=\sum^M_{i=0}\beta_i b_i(t)
\]
con coefficienti $\beta_i\in H^1_0(\Omega)$ e $b_i\in C([0,T]), b_i(t_j)=\delta_{ij}$ per $i,j=0,\dots,M$, l'equazione~\eqref{eqn:agg} diventa equivalente a
\begin{enumerate}
\item $\beta_M=0$,
\item trovare $\beta_i,i=0,\dots,M-1$ tale che
\begin{equation}
\label{eqn:agg1}
\frac{1}{t_{i+1}-t_i}(\beta_i-\beta_{i+1},g)_{L^2(\Omega)} + \frac{1}{2}a(\beta_i+\beta_{i+1},g)=\frac{1}{2}(h(t_i)+h(t_{i+1}),g)_{L^2(\Omega)}\quad\forall g\in H^1_0(\Omega)
\end{equation}
\end{enumerate}
Stabilità e convergenza:
\begin{lemma}
\label{stab:agg}
Sia $p_k\in P_k$ la soluzione di~\eqref{eqn:agg} con $h\in L^2(I,L^2(\Omega))$. Allora esiste una costante $C>0$ indipendente da $k$ tale che
\[
\norma{p_k}_{H^1(I,L^2(\Omega))} + \norma{p_k(0)}_{H^1(\Omega)} \le C\norma{h}_I.
\]
\end{lemma}
\begin{lemma}
\label{conv:agg}
Siano $p,p_k$ le soluzioni di \ref{eq:217} e di~\eqref{eqn:agg} rispettivamente, con $h\in L^2(I,L^2(\Omega))$. Allora vale
\[
\norma{p_k-p}_I\le Ck^2(\norma{\partial^2_t p}_I + \norma{\partial_t\Delta p}_I).
\]
\end{lemma}