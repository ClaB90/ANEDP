\chapter{Descrizione Implementazione}
\label{chap:Code}

In questo capitolo la struttura di base e le funzioni principali implementate verranno descritte.

\subsubsection{GitHub} 
\href{https://github.com/}{\texttt{GitHub}} è un sistema di controllo versione utilizzabile direttamente da linea di comando, molto diffuso e utile per tenere traccia delle varie fasi di sviluppo del codice. \texttt{GitHub} gestisce in modo adeguato i contributi al codice provenienti da agenti esterni e permette la condivisione del codice.\\
Il codice del progetto è reperibile su \texttt{GitHub} ed è possibile scaricarlo e collabolare allo sviluppo clonando il codice dalla repository  \href{https://github.com/}{\texttt{GitHub}}:\\
\begin{center}
\texttt{ git clone https://github.com/ClaB90/Progetto\_ANEDP}
\end{center}
Nella cartella principale è contenuto anche un file \texttt{.gitignore}, in cui sono specificate le estensioni dei files e le sottocartelle che non devono essere visionati in una repository \texttt{GitHub}. In particolare non si è interessati ai files temporanei che vengono eventualmente generati dagli editor.