\chapter{Descrizione Implementazione}
\label{chap:Code}

In questo capitolo la struttura di base e le funzioni principali del codice sviluppato verranno descritte.
Per l'implementazione del codice si è utilizzato il solver \href{http://www.freefem.org/}\texttt{FreeFem$++$}\footnote{http://www.freefem.org/} ed il relativo linguaggio di programazzione.

\subsubsection{\texttt{FreeFem$++$}}
\href{http://www.freefem.org/}\texttt{FreeFem$++$} è un software per la risoluzione di equazioni alle derivate parziali. Ha il proprio linguaggio di scripting basato sul C$++$. Gli script permettono di risolvere sistemi non lineari di più variabili in un dominio 2D o 3D.
\texttt{FreeFem$++$} è un software libero disponibile per i sistemi operativi \textit{Linux}, \textit{Solaris}, \textit{OS X} and \textit{MS Windows}.

\subsubsection{\texttt{GitHub}}
\href{https://github.com/}{\texttt{GitHub}}\footnote{https://github.com/} è un sistema di controllo versione utilizzabile direttamente da linea di comando, molto diffuso e utile per tenere traccia delle varie fasi di sviluppo del codice. \texttt{GitHub} gestisce in modo adeguato i contributi al codice provenienti da agenti esterni e permette la condivisione del codice.\\
Il codice del progetto è reperibile su \texttt{GitHub} ed è possibile scaricarlo e collabolare allo sviluppo clonando il codice dalla repository  \href{https://github.com/}{\texttt{GitHub}}:\\
\begin{center}
\texttt{ git clone https://github.com/ClaB90/Progetto\_ANEDP}
\end{center}
Nella cartella principale è contenuto anche un file \texttt{.gitignore}, in cui sono specificate le estensioni dei files e le sottocartelle che non devono essere visionati in una repository \texttt{GitHub}. In particolare non si è interessati ai files temporanei che vengono eventualmente generati dagli editor.