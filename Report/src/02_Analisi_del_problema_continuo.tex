\chapter{Analisi del problema continuo}
\label{chap:Continuos}

Introdotto l'intervallo temporale $I = (0,T) \subset \mathbb{R}$, $T < \infty$ e una funzione scelta $y_d \in L^2(I,L^2)$ consideriamo dunque il seguente problema di controllo ottimo lineare quadratico:
{\renewcommand\arraystretch{2}
\begin{equation}
\tag{$\mathbb{P}$}
\begin{aligned}
& \underset{y \in Y, u \in U_{ad}}{\text{min}}
& & J(y,u) = \frac{1}{2}{||y-y_d||^{2}}_{L^2(I,L^2(\Omega))} + \frac{\alpha}{2}{||u||^{2}}_U \\
& \text{s.t.} & &  y = S(Bu,y_0) \\
\label{eq:200}
\end{aligned}
\end{equation}
} %chiude \renewcommand\arraystretch{1.5}
Lo spazio dello stato Y è definito come:
\begin{equation}
Y = W(I) =  \left\{ v \in L^2(I, {H^0}_1(\Omega)), {{\partial}_{t}}v \in L^2(I, H^{-1}(\Omega)) \right\}
\label{eq:201}
\end{equation}
dove
\begin{equation}
Y \hookrightarrow C(\left[0,T\right], L^2(\Omega))
\label{eq:202}
\end{equation}