\chapter{Analisi del problema continuo}
\label{chap:Continuos}

In questo studio vengono considerati un dominio poligonale convesso $\Omega \in \mathbb{R}^n$ dove $n=2,3$, il cui bordo viene indicato con $\partial\Omega$, ed un intervallo temporale $I = (0,T) \subset \mathbb{R}$, $T < \infty$. 
Per l'analisi seguente viene introdotta la terna hilbertiana (${H^{0}_{1}(\Omega)},{L^{2}(\Omega)},{H^{-1}(\Omega)}$).
Il problema di controllo ottimo lineare quadratico analizzato è defnito come:
{\renewcommand\arraystretch{2}
\begin{equation}
\tag{$\mathbb{P}$}
\begin{aligned}
& \underset{y \in Y, u \in U_{ad}}{\text{min}}
& & J(y,u) = \frac{1}{2}{||y-y_d||^{2}}_{L^2(I,{L^{2}(\Omega)})} + \frac{\alpha}{2}{||u||^{2}}_U \\
& \text{s.t.} & & y = S(Bu,y_0) \\
\label{eq:200}
\end{aligned}
\end{equation}
} %chiude \renewcommand\arraystretch{1.5}
dove $y_d$ è una funzione scelta $\in L^2(I,{L^{2}(\Omega)})$.

\subsubsection{Problema di Stato}
Il problema di stato è definito in forma forte e in forma debole rispettivamente in \ref{eq:201} e \ref{eq:202}.
\begin{equation}
\begin{array}{cc}
 	{\partial_{t}}y - {\bigtriangleup}y = f & \text{in I}\times\Omega \\
	y=0 & \text{in I}\times\Omega \\
	y(0) = \kappa & \text{in }\Omega \\
\end{array}
\label{eq:201}
\end{equation}
$\text{?} y \in W(I) \text{ con } y(0)=\kappa \text{ e } \text{con (f,}\kappa) \in L^2(I,{H^{-1}(\Omega)})\times {L^{2}(\Omega)}$ \\
tale che:
%{\renewcommand\arraystretch{1.4}
\begin{equation}
\begin{array}{c}
	\int_{0}^{T} \left \langle {\partial_{t}}y(t),v(t) \right \rangle_{{H^{-1}(\Omega)}{H^{0}_{1}(\Omega)}} \, dt +  	\int_{0}^{T} a(y(t),v(t)) \, dt  \\
	 = \\
	\int_{0}^{T} \left \langle f(t),v(t) \right \rangle_{{H^{-1}(\Omega)}{H^{0}_{1}(\Omega)}} \, dt \ \ \forall v \in L^2(I,{H^{0}_{1}(\Omega)}) \\
\end{array}
\label{eq:202}
\end{equation}
%}
dove $y(t) e v(T) \in {H^{0}_{1}(\Omega)}$ e la forma bilineare $a(y(t),v(t)): {H^{0}_{1}(\Omega)}{\times}{H^{0}_{1}(\Omega)}\rightarrow\mathbb{R}$ è definita come:
\begin{equation}
 a(y,v) = \int_{\Omega} {\bigtriangledown}y(x){\bigtriangledown}v(x) \, dx
\label{eq:203}
\end{equation}
Lo spazio dello stato Y è definito come:
\begin{equation}
Y = W(I) =  \left\{ v \in L^2(I, {H^{0}_{1}(\Omega)}), {{\partial}_{t}}v \in L^2(I, {H^{-1}(\Omega)}) \right\}
\label{eq:204}
\end{equation}
ed in particolare vale che:
\begin{equation}
Y \hookrightarrow C(\left[0,T\right], {L^{2}(\Omega)})
\label{eq:205}
\end{equation}
l'operatore associato alla soluzione debole di \ref{eq:204} è  
\begin{equation}
S : L^2(I,{H^{-1}(\Omega)}) \times {L^{2}(\Omega)} \rightarrow Y \text{, } (f,\kappa) \longmapsto y = S(f,\kappa)
\label{eq:S}
\end{equation}
Applicando l'integrazione per parti sul \ref{eq:202} si ricava che:
\begin{equation}
A(y,v) = \int_{0}^{T} \left \langle f(t),v(t) \right \rangle_{{H^{-1}(\Omega)}{H^{0}_{1}(\Omega)}} \, dt + ({\kappa},v(0))_{L^{2}(\Omega)}
\label{eq:206}
\end{equation}
dove $y{\in}Y$ è la soluzione di \ref{eq:202}, $v{\in}Y$ è la funzione test e la forma bilineare $A(y,v): Y{\times}Y\rightarrow\mathbb{R}$ è definita come:
\begin{equation}
 A(y,v) = \int_{0}^{T} -\left \langle {\partial_{t}}v(t),y(t) \right \rangle_{{H^{-1}(\Omega)}{H^{0}_{1}(\Omega)}} \, dt + \int_{0}^{T} a(y(t),v(t)) \, dt + (y(T),v(T))_{L^{2}(\Omega)}
\label{eq:207}
\end{equation}
Per i risultati di stabilità, la consistenza e la convergenza di \ref{eq:201},noti in letteratura, si definisce y come soluzione unica di \ref{eq:206}. L'equazione \ref{eq:207} è necessaria per la definire  lo schema di approsimazione numerica per l'equazione di stato come descritto nel Capitolo \ref{chap:Discontinuos}.

\subsubsection{Spazio del Controllo}
Nello scenario descritto precedentemente la scelta per lo spazio di controllo non è unica. Seguendo le linee guida di \cite{MAIN} questo viene difinito come $U = L^2(I,\mathbb{R}^d), d \in \mathbb{N}$. Presi dunque $a_i, b_i \in \mathbb{R}$ t.c. $a_i<b_i {\forall}i=1:d$ la regione ammissibile, costituita da un insieme chiuso e convesso, è definita come:
\begin{equation}
U_{ad} = \left\{ u \in U | a_i \leq u_i(t) \geq b_i {\forall}i=1:d  \right\}
\label{eq:208}
\end{equation}
in questo caso, introdotti i funzionali $g_i \in {H^{-1}(\Omega)}$ l'operatore di controllo B, lineare e limitato, è definito da \ref{eq:209}.
\begin{equation}
B : U \rightarrow L^2(I,{H^{-1}(\Omega)}), u\mapsto \left( t\mapsto\sum_{i=1}^d u_i(t)g_i \right)
\label{eq:209}
\end{equation}
Si nota che l'operatore di controllo B può essere sostituito con l'operatore lineare affine $\tilde{B}$ definito come:
\begin{equation}
\tilde{B} : U \rightarrow L^2(I,{H^{-1}(\Omega)}) \text{, } u\mapsto g_0 + Bu
\label{eq:210}
\end{equation}
Affinchè non si perda la validità dei risultati che verranno esposti in seguito si suppone $g_0 \in L^2(I,{L^{2}(\Omega)})$ ed $g_0(0) \in {H^{0}_{1}(\Omega)}$
\MakeUppercase{è} quindi possibile introdurre l'operatore di proiezione ortogonale 
\begin{equation}
P_{U_{ad}} : L^2(I,\mathbb{R}^d)\rightarrow U_{ad}
\label{eq:211}
\end{equation}

\subsubsection{Problema Aggiunto}
Il problema \ref{eq:200} ammette un unica soluzione $(\overline{y},\overline{u}){\in}Y{\times}U$ dove $\overline{y}=S(B\overline{u},y_0)$.
Intodotti ora la variabile aggiunta $(\overline{p},\overline{q}) \in L^2(I,{H^{0}_{1}(\Omega)}{\times}{L^{2}(\Omega)})$, soluzione unica di \ref{eq:214}, e l'operatore aggiunto $B':L^2(I,{H^{0}_{1}(\Omega)}){\rightarrow}L^2(I,\mathbb{R}^d)$ definito da:
\begin{equation}
B'q(t) = ( \left \langle g_1,q(t) \right \rangle_{{H^{-1}(\Omega)}{H^{0}_{1}(\Omega)}}, . . . ,\left \langle g_d,q(t) \right \rangle_{{H^{-1}(\Omega)}{H^{0}_{1}(\Omega)}})^T
\label{eq:212}
\end{equation}
Il controllo ottimo, utilizzando l'operatore di proiezione ortogonale \ref{eq:211}, è caratterizatto dalla seguente condizione necessaria e suffciente di prim ordine:
\begin{equation}
\overline{u}=P_{U_{ad}}\left( -\frac{1}{\alpha}B'\overline{p} \right)
\label{eq:213}
\end{equation}
\begin{equation}
\begin{array}{c}
	\int_{0}^{T} \left \langle {\partial_{t}}\tilde{y}(t),\overline{p}(t) \right \rangle_{{H^{-1}(\Omega)}{H^{0}_{1}(\Omega)}} \, dt +  	\int_{0}^{T} a(\tilde{y}(t),\overline{p}(t)) \, dt + (\tilde{y}(0),\overline{q})_{L^{2}(\Omega)}  \\
	 = \\
	\int_{0}^{T} \int_{\Omega} (\overline{y}(t,x)-y_d(t,x))\tilde{y}(t,x) \,dx \, dt  \ \ \forall \tilde{y} \in Y \\
\end{array}
\label{eq:214}
\end{equation}
Si nota che per $v \in L^2(I,\mathbb{R}^d)$ vale che:
\begin{equation}
P_{U_{ad}}(v)(t) = {(P_{[a_i,b_i]}(v_i(t)))_{i=1}}^d
\label{eq:215}
\end{equation}
considerati $a,b,z \in \mathbb{R}$ $P_{[a,b]}(z) = max(a,min(z,b))$.
Poichè $\overline{y} - y_d \in L^2(I,{L^{2}(\Omega)})$ in \ref{eq:214}, si ha $\overline{p} \in Y$ ed integrando per parti con funzione di Y si trova che:
{\renewcommand\arraystretch{2}
\begin{equation}
\begin{array}{c}
	\int_{0}^{T} \left \langle -{\partial_{t}}\overline{p}(t),\tilde{y}(t) \right \rangle_{{H^{-1}(\Omega)}{H^{0}_{1}(\Omega)}} \, dt +  	\int_{0}^{T} a(\tilde{y}(t),\overline{p}(t)) \, dt \\
	+ (\tilde{y}(0),\overline{q})_{L^{2}(\Omega)} + (\tilde{y}(T),\overline{p}(T))_{L^{2}(\Omega)} - (\tilde{y}(0),\overline{p}(0))_{L^{2}(\Omega)} \\
	 = \\
	\int_{0}^{T} \int_{\Omega} (\overline{y}(t,x)-y_d(t,x))\tilde{y}(t,x) \,dx \, dt  \ \ \forall \tilde{y} \in Y \\
\end{array}
\label{eq:216}
\end{equation}
} %chiude arraystrech
Il problema \ref{eq:216} può essere riscritto in forma forte come:
\begin{equation}
\begin{aligned}
& {\partial_{t}}\overline{p} -\bigtriangleup\overline{p} = h & & \text{in }I{\times}\Omega \\
& \overline{p}=0 & & \text{in }{\times}\partial\Omega \\
& \overline{p}(T) & & \text{su }\Omega \\
\label{eq:217}
\end{aligned}
\end{equation}
dove $ h= \overline{y} - y_d$ e $\overline{q}=\overline{p}(0)$.

\subsubsection{Regolarità}
Per lo studio della regolarità di \ref{eq:201} e \ref{eq:215} è importante assumere che:
\begin{itemize}
\item[i)] $y_d \in H^1(I,{L^{2}(\Omega)})$ e $y_d(T) \in {H^{0}_{1}(\Omega)}$ ed $g_i \in {H^{0}_{1}(\Omega)} {\forall}i=1:d$ e $y_0 \in {H^{0}_{1}(\Omega)}$ con ${\bigtriangleup}y_0 \in {H^{0}_{1}(\Omega)}$.
\end{itemize} 
Per le dimostrazioni dei risultati esposti in questa sezione si rimanda a \cite{MAIN} o \cite{MV11}