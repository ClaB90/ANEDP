\documentclass{book}

\usepackage [utf8]{inputenc}
\usepackage[babel]{csquotes}
\usepackage [italian,english]{babel}					%Lingua e sillabazione
%\usepackage [english]{babel}
\usepackage {graphicx}									%Immagini
\usepackage {subfigure}
\usepackage {caption}									%Didascalie
\usepackage {wrapfig}									%Testo che avvolge un'oggetto
\usepackage {amsfonts, amsmath, amssymb, esint, mathrsfs}	%Font e simboli matematici
\usepackage {amsthm}										%Definizioni e teoremi
\usepackage {mathtools}									%Comandi abs e norm	definiti sotto
\usepackage {fancyhdr}									%Abstract in {book}
\usepackage {hyperref}									%Collegamenti rossi e verdi
\usepackage{braket}										%generate bra and ket vectors

% if you use this label name will show near by table and figure
%\usepackage {showlabels}

%should remove warning for textbraceleft
\usepackage[T1]{fontenc}

%Tabelle
\usepackage {booktabs}									
\usepackage {multirow}

%Per il typesetting di formule chimiche - comando \ce{formula}
%\usepackage {mhchem}	
										
%\usepackage {siunitx}

%Bib
\usepackage[citestyle=alphabetic,
            bibstyle=alphabetic,
            maxcitenames=1,
            hyperref,
            isbn=false,
            url=false,
            doi=false,
            eprint=false,
            firstinits=true,
            backend=biber]{biblatex}


% ti permette di usare \begin(comment) \end(comment)
\usepackage{verbatim}

%%%%%%%%%%%% Font Alternativi, a gruppi %%%%%%%%%%%%%%%%%%%
% \usepackage{mathpazo}
% \usepackage[scaled=.95]{helvet}
% \usepackage{courier}

% \usepackage{mathptmx}
% \usepackage[scaled=.90]{helvet}
% \usepackage{courier}

%\usepackage{fourier}

\newcommand{\mail}[1]{\href{mailto:#1}{\texttt{#1}}}

\usepackage {listings}						%Per il codice
	\lstset {
		language=C++,
%		inputencoding=utf8x,
%		frame=single,
		numbers=none,
		extendedchars=\true,
		showstringspaces=\false,
		basicstyle=\small,
		%Voglio i colori come li fa gedit!!
		keywordstyle=\color{darkgreen}\bfseries,
		commentstyle=\color{blue},
		stringstyle=\color{fuxia},
	}

\usepackage {color}							%Definire colori
	\ifx\color@rgb\@undefined\else
		\definecolor{darkgreen}{rgb}{0,0.55,0}
		\definecolor{fuxia}{rgb}{1,0,0.98}
	\fi



\makeatletter								%Gestione dei bookmark nel pdf
\newcommand\deftoclevel[2][chapter]{%
\expandafter\renewcommand\csname toclevel@#1\endcsname{#2}}
\makeatother		%il comando \deftoclevel{-1} riporta il bookmark fuori di un livello

% \usepackage {fancyhdr}						%Testatine ed Abstract in {book} come nella tesi liceo
% \fancyhf{}
% \fancyhead[RO,LE]{\thepage}
% \fancyhead[LO]{\nouppercase{\slshape\rightmark}}
% \fancyhead[RE]{\nouppercase{\slshape\leftmark}}
% \pagestyle{fancy}

\renewcommand{\epsilon}{\varepsilon}		%Definizione delle lettere greche per la tipografia italiana
\renewcommand{\theta}{\vartheta}
%\renewcommand{\rho}{\varrho}
\renewcommand{\phi}{\varphi}

\DeclareMathOperator{\NN}{\mathbb{N}}
\DeclareMathOperator{\RR}{\mathbb{R}}
\DeclareMathOperator\grad{grad}
\DeclareMathOperator\rot{rot}	
\DeclareMathOperator\curl{curl}	
\DeclareMathOperator\sym{sym}	
\DeclareMathOperator\tr{tr}	
\DeclareMathOperator\divergence{div}
\renewcommand{\div}{\divergence}

\theoremstyle{plain}
\newtheorem{lemma}{Lemma}

\DeclarePairedDelimiter{\abs}{\lvert}{\rvert}
\DeclarePairedDelimiter{\norma}{\lVert}{\rVert}

\def\d{\,\,\textrm{d}}				%il differenziale in testo dritto


%\def\wto{\overset{w}{\to}}
\def\wto{\rightharpoonup}
\def\vec{\mathbf}

\relpenalty=9999				%evita che le formule inline siano spezzate da fineriga QUASI sempre
\binoppenalty=9999				%impostare a 10000 per evitare SEMPRE

\newcommand{\<}{\langle}
\renewcommand{\>}{\rangle}

\newcommand{\numberset}{\mathbb}
\newcommand{\N}{\numberset{N}}
\newcommand{\R}{\numberset{R}}
\newcommand{\Z}{\numberset{Z}}

%Dichiarazione di teoremi e definizioni
\theoremstyle{definition}					
\newtheorem{definizione}{Definizione}

\theoremstyle{plain}
\newtheorem{teorema}{Teorema}
\newtheorem*{teorema_bis} {Teorema}
\newtheorem{proposizione}{Proposizione}

%%Bibliography
\bibliography{../src/library}
%\addbibresource{src/library.bib}

\usepackage{frontespizio}

\begin{document}
  
  \frontmatter
  \begin{titlepage}
    
\begin{frontespizio}

\Istituzione{POLITECNICO DI MILANO}
\Logo{img/polimiLogo}
\Facolta{Ingegneria Industriale e dell'informazione}
\Corso[Laurea]{Ingegneria Matematica}
\Annoaccademico{2015-2016}
\Sottotitolo{ sottotitolo}
\Titolo{\color[rgb]{.6,0,0} Strategie e analisi dell'errore per problemi di ottimizzazione vincolati regolati da equazioni di evoluzione}
\Candidato[]{Claudia Bonomi matr. 804378}
\Candidato[]{Edoardo Arbib matr. }
\Titoletto{Progetto per il corso di Analisi Numerica per le Equazioni a Derivate Parziali II}
\Relatore{ Simona Perotto}
\Relatore{ Ilario Mazzieri}
\Rientro{1cm}
\Margini{1.5cm}{2cm}{1.5cm}{3cm}

\end{frontespizio}
  \end{titlepage}
%    \input{src/quotes}
    \newcommand {\fncyblank}{\fancyhf{}}

\newenvironment {abstract}%
{%\fncyblank 
\null \vfill \begin {center}%
\bfseries \abstractname \end {center}}%
{\vfill \null}


\begin{abstract}

\end{abstract}

\clearpage


%\begin{otherlanguage}{italian}
%  \begin{abstract}
%    Questo e` un lavoro figo.
%  \end{abstract}
%\end{otherlanguage}

%\fancyhf{}
%\fancyhead[RO,LE]{\thepage}
%\fancyhead[LO]{\nouppercase{\slshape\rightmark}}
%\fancyhead[RE]{\nouppercase{\slshape\leftmark}}
%\pagestyle{fancy}

%   \chapter*{Acknowledgements}


%    \chapter*{Nomenclature and Acronyms}

\subsubsection{A}

\begin{tabular}{l l l l}
\end{tabular}

\subsubsection{B}

\begin{tabular}{l l l l}
\end{tabular}

\subsubsection{C}

\begin{tabular}{l l l l}
\end{tabular}

\subsubsection{D}

\begin{tabular}{l l l l}
\end{tabular}

\subsubsection{E}

\begin{tabular}{l l l l}
\end{tabular}

\subsubsection{F}

\begin{tabular}{l l l l}
\end{tabular}

\subsubsection{G}

\begin{tabular}{l l l l}
\end{tabular}

\subsubsection{H}

\begin{tabular}{l l l l}
\end{tabular}

\subsubsection{I}

\begin{tabular}{l l l l}
\end{tabular}

\subsubsection{J}

\begin{tabular}{l l l l}
\end{tabular}

\subsubsection{K}

\begin{tabular}{l l l l}
\end{tabular}

\subsubsection{L}

\begin{tabular}{l l l l}
\end{tabular}

\subsubsection{M}

\begin{tabular}{l l l l}
\end{tabular}

\subsubsection{N}

\begin{tabular}{l l l l}
\end{tabular}

\subsubsection{O}

\begin{tabular}{l l l l}
\end{tabular}

\subsubsection{P}

\begin{tabular}{l l l l}
\end{tabular}

\subsubsection{Q}

\begin{tabular}{l l l l}
\end{tabular}

\subsubsection{R}

\begin{tabular}{l l l l}
\end{tabular}

\subsubsection{S}

\begin{tabular}{l l l l}
\end{tabular}

\subsubsection{T}

\begin{tabular}{l l l l}
\end{tabular}

\subsubsection{U}

\begin{tabular}{l l l l}
\end{tabular}

\subsubsection{V}

\begin{tabular}{l l l l}
\end{tabular}

\subsubsection{W}

\begin{tabular}{l l l l}
\end{tabular}

\subsubsection{X}

\begin{tabular}{l l l l}
\end{tabular}

\subsubsection{Y}

\begin{tabular}{l l l l}
\end{tabular}

\subsubsection{Z}

\begin{tabular}{l l l l}
\end{tabular}
%    \subsubsection{\textbf{Greek Symbols}}


\subsubsection{$ \mu $}

\begin{tabular}{l l l l}
\end{tabular}

\subsubsection{$ \nu $}

\begin{tabular}{l l l l}
\end{tabular}


\subsubsection{$ \Omega $}

\begin{tabular}{l l l l}
\end{tabular}
  \tableofcontents
  \mainmatter
    \chapter{Introduzione}

Il lavoro qui presentato tratta lo studio di un problema di controllo ottimo parabolico attraverso l'analisi proposta da \cite{MAIN}.
\par\medskip
Per l'equazione di stato in tempo viene utilizzato uno schema Petrov-Galerkin con un approccio costante a tratti per la funzione di stato ed uno lineare a tratti per la funzione test. Questa scelta degli spazi funzionali ha una ripercussione sullo schema di discretizzazione temporale sia del problema di stato che del problema aggiunto. Per entrambi, infatti, sarà utilizzata una variante dello schema di Crank-Nicolson consistente con la teoria di Rannacher descritta in \cite{Ran84}.
In \cite{MAIN} viene provato analaticamente che questa scelta permette di raggiungere un ordine due di convergenza temporale sia per l'errore di controllo che per l'errore dello stato proiettato sulla griglia duale.
Per la discretizazione spaziale si è fatto riferimento all'analisi proposta in \cite{MV11}.
\par\medskip
Attraverso l'utilizzo del software \textbf{FreeFem++} l'approccio teorico proposto precedentemente è stato implementato. I risultati numerici ottenuti confermano quelli teorici e sono consistenti con quelli presentati in \cite{MAIN}. Per il calcolo dell'errore di controllo è stato utilizzato inizialmente il metodo di Cavalieri-Simpson. In seguito è stato calcolato un secondo algoritmo meno soggetto agli errori di approssimazione, con esso si trova unordine di convergenza maggiore di 2 per l'errore di controllo.
\par\medskip
Il report è strutturato nel seguente modo. Nel Capitolo \ref{chap:Continuos} viene analizzata la soluzione teorica del problema di controllo ottimo ed introdotti i risulati di regolarita per l'equazione di stato e per l'equazione aggiunta. Nel Capitolo \ref{chap:Discontinuos} viene analizzata la regolità del problema discontinuo e introdotte la semi-discretizzazione temporale e la discretizzazione spaziale. Nel Capitolo \ref{chap:Code} sono contenute le informazioni riguardanti l'implementazione dell'algoritmo. Nel Capitolo \ref{chap:Results} sono raccolti i risultati numerici. Nel Capitolo \ref{chap:Conclusion} sono contenute le conclusioni e spunti per lavori futuri.
    \chapter{Analisi del problema continuo}
\label{chap:Continuos}

In questo studio vengono considerati un dominio poligonale convesso $\Omega \in \mathbb{R}^n$ dove $n=2,3$, il cui bordo viene indicato con $\partial\Omega$, ed un intervallo temporale $I = (0,T) \subset \mathbb{R}$, $T < \infty$. 
Per l'analisi seguente viene introdotta la terna hilbertiana (V,H,$V^*$) dove $V={H^0}_1(\Omega)$, $H=L^2(\Omega)$ e $V^*=H^{-1}(\Omega)$.
Il problema di controllo ottimo lineare quadratico analizzato è defnito come:
{\renewcommand\arraystretch{2}
\begin{equation}
\tag{$\mathbb{P}$}
\begin{aligned}
& \underset{y \in Y, u \in U_{ad}}{\text{min}}
& & J(y,u) = \frac{1}{2}{||y-y_d||^{2}}_{L^2(I,H)} + \frac{\alpha}{2}{||u||^{2}}_U \\
& \text{s.t.} & & y = S(Bu,y_0) \\
\label{eq:200}
\end{aligned}
\end{equation}
} %chiude \renewcommand\arraystretch{1.5}
dove $y_d$ è una funzione scelta $\in L^2(I,H)$.

\subsubsection{Problema di Stato}
Il problema di stato è definito in forma forte e in forma debole rispettivamente in \ref{eq:201} e \ref{eq:202}.
\begin{equation}
\begin{array}{cc}
 	{\partial_{t}}y - {\bigtriangleup}y = f & \text{in I}\times\Omega \\
	y=0 & \text{in I}\times\Omega \\
	y(0) = \kappa & \text{in }\Omega \\
\end{array}
\label{eq:201}
\end{equation}
$\text{?} y \in W(I) \text{ con } y(0)=\kappa \text{ e } \text{con (f,}\kappa) \in L^2(I,V^*)\times H$ \\
tale che:
%{\renewcommand\arraystretch{1.4}
\begin{equation}
\begin{array}{c}
	\int_{0}^{T} \left \langle {\partial_{t}}y(t),v(t) \right \rangle_{V^*V} \, dt +  	\int_{0}^{T} a(y(t),v(t)) \, dt  \\
	 = \\
	\int_{0}^{T} \left \langle f(t),v(t) \right \rangle_{V^*V} \, dt \ \ \forall v \in L^2(I,V) \\
\end{array}
\label{eq:202}
\end{equation}
%}
dove $y(t) e v(T) \in V$ e la forma bilineare $a(y(t),v(t)): V{\times}V\rightarrow\mathbb{R}$ è definita come:
\begin{equation}
 a(y,v) = \int_{\Omega} {\bigtriangledown}y(x){\bigtriangledown}v(x) \, dx
\label{eq:203}
\end{equation}
Lo spazio dello stato Y è definito come:
\begin{equation}
Y = W(I) =  \left\{ v \in L^2(I, V), {{\partial}_{t}}v \in L^2(I, V^*) \right\}
\label{eq:204}
\end{equation}
ed in particolare vale che:
\begin{equation}
Y \hookrightarrow C(\left[0,T\right], H)
\label{eq:205}
\end{equation}
l'operatore associato alla soluzione debole di \ref{eq:204} è $S : L^2(I,V^*) \times H \rightarrow Y$, $(f,\kappa) \longmapsto y = S(f,\kappa)$ 
\par
Applicando l'integrazione per parti sul \ref{eq:202} si ricava che:
\begin{equation}
A(y,v) = \int_{0}^{T} \left \langle f(t),v(t) \right \rangle_{V^*V} \, dt + ({\kappa},v(0))_H
\label{eq:206}
\end{equation}
dove $y{\in}Y$ è la soluzione di \ref{eq:202}, $v{\in}Y$ è la funzione test e la forma bilineare $A(y,v): Y{\times}Y\rightarrow\mathbb{R}$ è definita come:
\begin{equation}
 A(y,v) = \int_{0}^{T} -\left \langle {\partial_{t}}v(t),y(t) \right \rangle_{V^*V} \, dt + \int_{0}^{T} a(y(t),v(t)) \, dt + (y(T),v(T))_H
\label{eq:207}
\end{equation}
Per i risultati di stabilità, la consistenza e la convergenza di \ref{eq:201},noti in letteratura, si definisce y come soluzione unica di \ref{eq:206}. L'equazione \ref{eq:207} è necessaria per la definire  lo schema di approsimazione numerica per l'equazione di stato come descritto nel Capitolo \ref{chap:Discontinuos}.

\subsubsection{Spazio del Controllo}
Nello scenario descritto precedentemente la scelta per lo spazio di controllo non è unica. Seguendo le linee guida di \cite{MAIN} questo viene difinito come $U = L^2(I,\mathbb{R}^d), d \in \mathbb{N}$. Presi dunque $a_i, b_i \in \mathbb{R}$ t.c. $a_i<b_i {\forall}i=1:d$ la regione ammissibile, costituita da un insieme chiuso e convesso, è definita come:
\begin{equation}
U_{ad} = \left\{ u \in U | a_i \leq u_i(t) \geq b_i {\forall}i=1:d  \right\}
\label{eq:208}
\end{equation}
in questo caso, introdotti i funzionali $g_i \in V^*$ l'operatore di controllo B, lineare e limitato, è definito da \ref{eq:209}.
\begin{equation}
B : U \rightarrow L^2(I,V^*), u\mapsto \left( t\mapsto\sum_{i=1}^d u_i(t)g_i \right)
\label{eq:209}
\end{equation}
Si nota che l'operatore di controllo B può essere sostituito con l'operatore lineare affine $\tilde{B}$ definito come:
\begin{equation}
\tilde{B} : U \rightarrow L^2(I,V^*) \text{, } u\mapsto g_0 + Bu
\label{eq:210}
\end{equation}
Affinchè non si perda la validità dei risultati che verranno esposti in seguito si suppone $g_0 \in L^2(I,H)$ ed $g_0(0) \in V$
\MakeUppercase{è} quindi possibile introdurre l'operatore di proiezione ortogonale 
\begin{equation}
P_{U_{ad}} : L^2(I,\mathbb{R}^d)\rightarrow U_{ad}
\label{eq:211}
\end{equation}

\subsubsection{Problema Aggiunto}
Il problema \ref{eq:200} ammette un unica soluzione $(\overline{y},\overline{u}){\in}Y{\times}U$ dove $\overline{y}=S(B\overline{u},y_0)$.
Intodotti ora la variabile aggiunta $(\overline{p},\overline{q}) \in L^2(I,V{\times}H)$, soluzione unica di \ref{eq:214}, e l'operatore aggiunto $B':L^2(I,V){\rightarrow}L^2(I,\mathbb{R}^d)$ definito da:
\begin{equation}
B'q(t) = ( \left \langle g_1,q(t) \right \rangle_{V*V}, . . . ,\left \langle g_d,q(t) \right \rangle_{V*V})^T
\label{eq:212}
\end{equation}
Il controllo ottimo, utilizzando l'operatore di proiezione ortogonale \ref{eq:211}, è caratterizatto dalla seguente condizione necessaria e suffciente di prim ordine:
\begin{equation}
\overline{u}=P_{U_{ad}}\left( -\frac{1}{\alpha}B'\overline{p} \right)
\label{eq:213}
\end{equation}
\begin{equation}
\begin{array}{c}
	\int_{0}^{T} \left \langle {\partial_{t}}\tilde{y}(t),\overline{p}(t) \right \rangle_{V^*V} \, dt +  	\int_{0}^{T} a(\tilde{y}(t),\overline{p}(t)) \, dt + (\tilde{y}(0),\overline{q})_H  \\
	 = \\
	\int_{0}^{T} \int_{\Omega} (\overline{y}(t,x)-y_d(t,x))\tilde{y}(t,x) \,dx \, dt  \ \ \forall \tilde{y} \in Y \\
\end{array}
\label{eq:214}
\end{equation}
Si nota che per $v \in L^2(I,\mathbb{R}^d)$ vale che:
\begin{equation}
P_{U_{ad}}(v)(t) = {(P_{[a_i,b_i]}(v_i(t)))_{i=1}}^d
\label{eq:215}
\end{equation}
dove considerati $a,b,z \in \mathbb{R}$ $P_{[a,b]}(z) = max(a,min(z,b))$.
Poichè $\overline{y} - y_d \in L^2(I,H)$ in \ref{eq:214}, si ha $\overline{p} \in Y$ ed integrando per parti con funzione di Y si trova che:
{\renewcommand\arraystretch{2}
\begin{equation}
\begin{array}{c}
	\int_{0}^{T} \left \langle -{\partial_{t}}\overline{p}(t),\tilde{y}(t) \right \rangle_{V^*V} \, dt +  	\int_{0}^{T} a(\tilde{y}(t),\overline{p}(t)) \, dt \\
	+ (\tilde{y}(0),\overline{q})_H + (\tilde{y}(T),\overline{p}(T))_H - (\tilde{y}(0),\overline{p}(0))_H \\
	 = \\
	\int_{0}^{T} \int_{\Omega} (\overline{y}(t,x)-y_d(t,x))\tilde{y}(t,x) \,dx \, dt  \ \ \forall \tilde{y} \in Y \\
\end{array}
\label{eq:216}
\end{equation}
} %chiude arraystrech
Il problema \ref{eq:216} può essere riscritto in forma forte come:
\begin{equation}
\begin{aligned}
& {\partial_{t}}\overline{p} -\bigtriangleup\overline{p} = h & & in I{\times}\Omega \\
& \overline{p}=0 & & in I{\times}\partial\Omega \\
& \overline{p}(T) & & su \Omega \\
\label{eq:217}
\end{aligned}
\end{equation}
dove $ h= \overline{y} - y_d$ e $\overline{q}=\overline{p}(0)$.

\subsubsection{Regolarità}
Per lo studio della regolarità di \ref{eq:201} e \ref{eq:215} è importante assumere che:
\begin{itemize}
\item[i)] $y_d \in H^1(I,H)$ e $y_d(T) \in V$ ed $g_i \in V {\forall}i=1:d$ e $y_0 \in V$ con ${\bigtriangleup}y_0 \in V$.
\end{itemize} 
Per le dimostrazioni dei risultati esposti in questa sezione si rimanda a \cite{MAIN} o \cite{MV11}
    \chapter{Analisi del problema discreto}

\section{Semi-discretizzazione temporale}

\section{Discretizzazione spaziale-temporale}
    \chapter{Descrizione Implementazione}
 	\chapter{Risultati numerici}
\label{chap:Results}

\section{Test Case 01}

\section{Test Case 02}   
 	\chapter{Conclusioni}
\label{chap:Conclusion}
    
  \backmatter
  
  \nocite{QUART}
  \printbibliography[heading=bibintoc]    
    
  \listoffigures
  \listoftables

\end{document}
